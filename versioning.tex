\section{Data versioning and archival}
Once an updated version of data with improved data quality is prepared,
it is formally assigned a version for tracking. However, versioning for
complex data like those in ARM pose unique challenges which led to
development of new data versioning system deployed across the facility
\cite{Macduff_2014}. Data reprocessing to data quality improvement,
however, at times leads to versioning conflict scenarios. Versioning
module of the workflow extends the current versioning scheme
\cite{Macduff_2014}.

Standard ARM filename and versioning includes site, facility, instrument
and start time of observation time series in filename. However, length
of time series contained within the file are not reflected in the
filename and thus creating conflict when data are reprocessed.
ARM file standards also allow for any arbitrary number of files within a day
during a real time processing, while reprocessing for data quality
improvements on past data often create longer continuous time series and
thus few files. To identify and address these conflicts all affected
files (old and new) were checked for the start and end of the time
series. In particular following scenarios occur frequently: 

\begin{enumerate}
 \item Near-real-time processing may create multiple partial day files
 within a day (ex. fileA.v0: 00:00:00--12:00:00 and fileB.v0:
 12:00:01--24:00:00) if the observations from the remote instruments as
 observations arrive at the data center in chunks 
 due to network bandwidth or some other issue. However, the reprocessing
 for the data may create a single file with continuous time series 
 (ex. fileA: 00:00:00--24:00:00).
 While the filename of the newly created file would be same as one of
 the existing files and will be assigned the correct incremental
 version number (ex. fileA.v1: 00:00:00--24:00:00), the rest of the 
 old files (ex. fileB: 12:00:01--24:00:00) from the day would identified
 and marked to be obsolete and marked unavailable in future.
 \item File versioning tracks the versions of a filename, but during the
 process of reprocessing it's possible for new filenames to be created.
 For example, if data in original fileA.v0: 01:00:00--24:00:00 was
 reprocessed to correct the quality issue during first hour of the day
 that was excluded for the original processing, a new fileB.v0:
 00:00:00--24:00:00 will be created and while no filename conflict occur
 with fileA.v0, fileA.v0 must be deleted and marked unavailable to avoid
 conflicting time series data.
\end{enumerate}

Once the file has been assigned the correct name and version, it is
archived in the deep archive on HPSS and a copy is placed on new data
cache while the older data are made unavailable. ARM's comprehensive
database of all file holdings are also updated as a part of the archival step.

