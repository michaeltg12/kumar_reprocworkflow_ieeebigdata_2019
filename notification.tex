\section{Communicating data quality changes}
Clear and timely communication of the data quality changes are important
for scientific users. While the description of data quality issues and
remedial actions taken to address them are available publicly as part of
the DQR associated with the data, they may not reach the data users who
have downloaded and used the data in the past. We utilized the database
of historical data download history to identify all users who have
downloaded the data affected as part of a data reprocessing and
communicate via an automated email to them a summary of changes to the
data to inform them of the change. The email notifications are also sent
to instrument principal investigators, and developer of Value Added
Product that are using the data product and thus may be affected by the
change in the data quality.


