\section{Introduction}
Rapid growth in observational technologies and data management and
sharing infrastructure has enabled a new era of big data enabled science
discovery in wide range of disciplines. However, ensuring the high
quality of the data is crucial for accurate and reliable scientific
research and decision making. Management of data quality especially
presents challenges for environmental monitoring networks that operate
large numbers of distributed facilities in remote regions of the world,
like U.S. Department of Energy's Atmospheric Radiation Measurement (ARM)
program (\url{https://arm.gov}). Ensuring high quality of data to
scientific community requires not only  the assessment and
documentation of data quality issues but also appropriate data reprocessing to
address and improve the data quality.
Equally important is to capture the provenance at every step of
the data life cycle that are comprehensive, timely and transparently
communicated to the data users.
Performing these data processing at the scale for scientific data center
like ARM pose a big data challenge. Diversity of sensors and instruments
call for an comprehensive metadata and data processing automation.
Volume of the data necessitates state-of-the-art parallel processing,
efficient data movement, I/O and provenance tracking and management.
In this paper we describe a provenance-aware workflow for
data quality improvement for continuously growing
atmospheric science data streams within ARM's Petascale archive.


