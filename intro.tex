\section{Introduction}
Rapid growth in observational technologies and data management and
sharing infrastructure has enabled a new era of big data enabled science
discovery in wide range of disciplines. However, ensuring the high
quality of the data is crucial for accurate and reliable scientific
research and decision making. Management of data quality especially
presents challenges for environmental monitoring networks that operate
large numbers of distributed facilities in remote regions of the world,
like U.S. Department of Energy's Atmospheric Radiation Measurement (ARM)
program (\url{https://arm.gov}). Ensuring high quality of data to
scientific community requires a framework to not only assess and
document any data quality issues but also perform appropriate data reprocessing to
address and improve the data quality.
Equally important is to capture the provenance at every step of
the data life cycle that are comprehensive, timely and transparently
communicated to the data users.
\hl{But its a big data problem.. diversity of instruments and data means
	automated intelligent processing schemes are needed, large data
		volume calls for parallel procesing, efficient I/O and data
		transfers, maintaining confidence in data, enabling
		reproducibility requires formal version tracking, provenance
		tracking}

In this paper we describe a provenance-aware workflow for
data quality improvement for continuously growing
atmospheric science data streams within ARM's petascale-archive.


